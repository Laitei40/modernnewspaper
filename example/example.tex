\documentclass{article}
\usepackage{modernnewspaper}

% ---------------------------------
% Metadata
% ---------------------------------
\SetPaperName{Modern Newspaper}
\SetPaperSlogan{Unicode Multilingual Test Edition}
\SetPaperLocation{Yangon, Myanmar}
\SetPaperWebsite{https://github.com/Laitei40/modernnewspaper}
\SetPaperVolume{1}
\SetPaperIssue{5}
\SetPaperDate{\today}

\begin{document}
	
	% =================================================
	% PAGE 1 — FRONT PAGE
	% =================================================
	\MakePaperHeader
	
	\BeginNewsColumns{3}
	
	% ---------------- Myanmar ----------------
	\begin{article}
		\headline{မြန်မာဘာသာ}
		\byline{Laitei}
		
		{\MyanmarFont
			\DropCap{မ}ျန်မာစာသည် ယူနီကုဒ်အခြေခံဖြင့် ရေးသားရသော ဘာသာစကားဖြစ်ပြီး
			ခေတ်မီ သတင်းစာပုံစံများတွင် မှန်ကန်စွာ ဖော်ပြနိုင်ရပါသည်။
		}
	\end{article}
	
	% ---------------- Arabic (RTL) ----------------
	\EnableRTL
	\SetByWord{بقلم}
	
	\begin{article}
		\headline{اللغة العربية}
		\byline{لايتي}
		
		{\ArabicFont
			\DropCap{ا}للغة العربية تُكتب من اليمين إلى اليسار،
			وهذا المثال يوضح دعم النصوص العربية في الأعمدة متعددة الصفحات.
		}
	\end{article}
	
	\DisableRTL
	\SetByWord{By}
	
	% ---------------- Hindi ----------------
	\begin{article}
		\headline{हिन्दी भाषा}
		\byline{Editor}
		
		{\DevaFont
			\DropCap{ह}िन्दी देवनागरी लिपि में लिखी जाने वाली भाषा है।
			यह उदाहरण यूनिकोड समर्थन को दर्शाता है।
		}
	\end{article}
	
	\EndNewsColumns
	
	% =================================================
	% PAGE BREAK
	% =================================================
	\newpage
	
	% =================================================
	% PAGE 2 — INNER PAGE (NEW HEADER)
	% =================================================
	

\vspace{0.1em}
\hrule
\vspace{0.1em}

\begin{center}
{\small Modern Newspaper — Vol. 2, No. 1 \quad | \quad \today}
\end{center}

\vspace{0.1em}
\hrule
\vspace{0.1em}


	
	\BeginNewsColumns{3}
	
	% ---------------- Chinese ----------------
	\begin{article}
		\headline{中文}
		\byline{编辑}
		
		{\CJKFont
			\DropCap{中}文在现代排版系统中对 Unicode 的支持非常重要。
			本页展示了报纸内页的多语言排版效果。
		}
	\end{article}
	
	% ---------------- English ----------------
	\begin{article}
		\headline{English Language}
		\byline{Editor}
		
		\DropCap{E}nglish now appears together with all other scripts correctly.
		This inner page intentionally uses a simplified header,
		which is typical in real-world newspaper layouts.
	\end{article}
	
	% ---------------- Mixed article ----------------
	\begin{article}
		\headline{Multilingual Publishing Today}
		\byline{Laitei}
		
		\DropCap{M}odern newspapers often publish content in multiple languages.
		A Unicode-first LaTeX package makes it possible to present
		Myanmar, Arabic, Hindi, Chinese, and English together
		without layout conflicts.
	\end{article}
	
	\EndNewsColumns
	
\end{document}
