\documentclass{article}

% Load the modern newspaper package
\usepackage{modernnewspaper}

% -------------------------------
% Newspaper metadata
% -------------------------------
\SetPaperName{Modern Newspaper}
\SetPaperSlogan{Informing and educating for tomorrow}
\SetPaperLocation{Yangon, Myanmar}
\SetPaperWebsite{https://github.com/laitei40/modernnewspaper}
\SetPaperVolume{1}
\SetPaperIssue{1}
\SetPaperDate{\today}

\begin{document}

% -------------------------------
% Masthead
% -------------------------------
\MakePaperHeader

% -------------------------------
% Newspaper content
% -------------------------------
\BeginNewsColumns{2}

\begin{article}
\headline{A Modern LaTeX Newspaper Package Is Born}
\byline{Laithon}

This is an example document demonstrating the
\textbf{modernnewspaper} package. The goal of this package is to
provide a clean, Unicode-first newspaper layout that works well for
both print and digital publications.

The package is designed for use with \textit{XeLaTeX} or
\textit{LuaLaTeX} and supports multilingual text out of the box.

Unicode test:

\medskip
\noindent
မြန်မာစာ · العربية · हिन्दी · 中文 · English
\end{article}

\begin{article}
\headline{Why a New Newspaper Package?}
\byline{Editor}

The traditional \texttt{newspaper} package is print-oriented and
predates modern Unicode workflows. This project introduces native
support for web metadata such as website URLs, modern typography, and
clean extensibility.

Future versions will include drop caps, image wrapping, RTL support,
themes, and improved accessibility.
\end{article}

\EndNewsColumns

\end{document}
